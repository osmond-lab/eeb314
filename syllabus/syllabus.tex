% Options for packages loaded elsewhere
\PassOptionsToPackage{unicode}{hyperref}
\PassOptionsToPackage{hyphens}{url}
%
\documentclass[
]{article}
\usepackage{amsmath,amssymb}
\usepackage{iftex}
\ifPDFTeX
  \usepackage[T1]{fontenc}
  \usepackage[utf8]{inputenc}
  \usepackage{textcomp} % provide euro and other symbols
\else % if luatex or xetex
  \usepackage{unicode-math} % this also loads fontspec
  \defaultfontfeatures{Scale=MatchLowercase}
  \defaultfontfeatures[\rmfamily]{Ligatures=TeX,Scale=1}
\fi
\usepackage{lmodern}
\ifPDFTeX\else
  % xetex/luatex font selection
\fi
% Use upquote if available, for straight quotes in verbatim environments
\IfFileExists{upquote.sty}{\usepackage{upquote}}{}
\IfFileExists{microtype.sty}{% use microtype if available
  \usepackage[]{microtype}
  \UseMicrotypeSet[protrusion]{basicmath} % disable protrusion for tt fonts
}{}
\makeatletter
\@ifundefined{KOMAClassName}{% if non-KOMA class
  \IfFileExists{parskip.sty}{%
    \usepackage{parskip}
  }{% else
    \setlength{\parindent}{0pt}
    \setlength{\parskip}{6pt plus 2pt minus 1pt}}
}{% if KOMA class
  \KOMAoptions{parskip=half}}
\makeatother
\usepackage{xcolor}
\setlength{\emergencystretch}{3em} % prevent overfull lines
\providecommand{\tightlist}{%
  \setlength{\itemsep}{0pt}\setlength{\parskip}{0pt}}
\setcounter{secnumdepth}{-\maxdimen} % remove section numbering
\usepackage{bookmark}
\IfFileExists{xurl.sty}{\usepackage{xurl}}{} % add URL line breaks if available
\urlstyle{same}
\hypersetup{
  hidelinks,
  pdfcreator={LaTeX via pandoc}}

\author{}
\date{}
\usepackage{fullpage}
\begin{document}

\section{Mathematical modelling in ecology and evolution (EEB314)}

Mathematics is central to science because it provides a rigorous way to
go from a set of assumptions to their logical consequences. In ecology
\& evolution this might be how we think a virus will spread and evolve,
how climate change will impact a threatened population, or how much
genetic diversity we expect to see in a randomly mating population. In
this course you'll learn how to build, analyze, and interpret
mathematical models of increasing complexity through readings, lectures,
computer labs, and a final project. Our focus is on deterministic
dynamical models (recursions and differential equations), which requires
us to learn and use some calculus and linear algebra.

Please see the
\href{https://artsci.calendar.utoronto.ca/course/eeb314h1}{University of
Toronto Academic Calendar} for more details on the course prerequisites
and additional information on the distribution/breadth requirements this
course satisfies.

Next taught: Fall 2025

Previously taught: Fall 2024, Fall 2022 (EEB430), Fall 2021 (EEB430)

\section{Instructors}\label{instructors}

\subsection{Professor}\label{professor}

Matthew Osmond (he/him)

\begin{itemize}
\tightlist
\item
  email: mm.osmond@utoronto.ca
\item
  website: \href{https://osmond-lab.github.io/}{osmond-lab.github.io}
\end{itemize}

\subsection{Teaching assistant}\label{teaching-assistant}

Erik Curtis (he/him)

\begin{itemize}
\tightlist
\item
  email: erik.curtis@mail.utoronto.ca
\end{itemize}

\section{When and where}\label{when-and-where}

\subsection{Lectures}\label{lectures}

\begin{itemize}
\tightlist
\item
  Monday \& Wednesday, 10:10 - 11:00 AM
\item
  Wilson Hall - New College (WI), room 523
\end{itemize}

\subsection{Labs}\label{labs}

\begin{itemize}
\tightlist
\item
  Wednesday, 3:10 - 5:00 PM
\item
  Sidney Smith (SS), room 561
\end{itemize}

\section{Course structure}\label{course-structure}

\subsection{Learning objectives}\label{learning-objectives}

Mathematics is central to science because it provides a rigorous way to
go from a set of assumptions (what we take to be true) to their logical
consequences (what we want to know). In ecology \& evolution this might
be how we think SARS-CoV-2 may spread and evolve given a set of
vaccination rates and travel restrictions, how caribou population sizes
are predicted to respond to forecasted rates of climate change, or
something much more abstract like the expected amount of genetic
diversity in a randomly mating population. In this course we'll learn
how to build, analyze, and interpret mathematical models of increasing
complexity through readings, lectures, computer labs, and a final
project. By the end of the course you will be able to:

\begin{itemize}
\tightlist
\item[$\boxtimes$]
  build a model: go from a verbal description of a biological system to
  a set of equations
\item[$\boxtimes$]
  analyze a model: manipulate a set of equations into a mathematical
  expression of interest
\item[$\boxtimes$]
  interpret a model: translate mathematical expressions back into
  biological meaning
\end{itemize}

\subsection{Weekly tasks}\label{weekly-tasks}

\begin{itemize}
\tightlist
\item[$\boxtimes$]
  read text
\item[$\boxtimes$]
  attend two lectures
\item[$\boxtimes$]
  attend one lab
\end{itemize}

\subsection{Grading scheme}\label{grading-scheme}

\begin{itemize}
\tightlist
\item
  in-class tests: 4 x 20\%
\item
  final project: 20\%
\end{itemize}

\section{Textbook}\label{textbook}

Otto \& Day 2007. \href{https://www.zoology.ubc.ca/biomath/}{A
biologist's guide to mathematical modeling in ecology and evolution}.

\begin{itemize}
\tightlist
\item
  \href{https://librarysearch.library.utoronto.ca/permalink/01UTORONTO_INST/14bjeso/alma991106921343406196}{UofT
  library e-copies}
\item
  \href{https://librarysearch.library.utoronto.ca/permalink/01UTORONTO_INST/14bjeso/alma991106624476006196}{UofT
  library physical-copies}
\item
  \href{https://press.princeton.edu/books/hardcover/9780691123448/a-biologists-guide-to-mathematical-modeling-in-ecology-and-evolution}{buy
  your own copy}
\end{itemize}

\section{Final project}\label{final-project}

\subsection{Construct your own model}\label{construct-your-own-model}

In this project you will use the tools you've learned in class and apply
them to a model that you develop. The model can be about any phenomenon
in ecology and evolution, as long as you make up the model.

You'll do the final project in two parts.

\begin{itemize}
\tightlist
\item[$\boxtimes$]
  Part 1

  \begin{itemize}
  \tightlist
  \item
    Describe your biological question and why this interests you
  \item
    Describe your model in words (ie, the main assumptions) and explain
    the main structure with a diagram (eg, flow or life cycle diagram)
  \item
    Write down the equations that you will analyze
  \item
    Describe what your analysis might reveal (ie, your hypothesis)
  \item
    Max 2 pages
  \item
    \href{final_project/partI_example.md}{Example}
  \end{itemize}
\item[$\boxtimes$]
  Part 2

  \begin{itemize}
  \tightlist
  \item
    Re-iterate your biological question and why this interests you
  \item
    Describe your model assumptions in detail, defining all parameters
    and variables
  \item
    Write down the equations for your model
  \item
    Analyze your model
  \item
    Explain how the results address your original question
  \item
    Suggest how the model could be improved or extended
  \item
    Max 4 pages (not including any code that you used, which can be
    included as a link or additional file)
  \item
    \href{final_project/partII_example.md}{Example}
  \end{itemize}
\end{itemize}

!!! tip

\begin{verbatim}
If you are having trouble coming up with a new model, take one of the models that we've analysed in the course and change one or more of its underlying assumptions to get a new set of equations. Then analyse these equations. Discuss the differences between the assumptions used and also  between the results obtained.
\end{verbatim}

\section{Exams}\label{exams}

\subsection{Previous midterms}\label{previous-midterms}

Roughly covers univariate lectures and labs.

\begin{itemize}
\tightlist
\item
  \href{exam_files/midterm2024.pdf}{2024},
  \href{exam_files/midterm2024_solns.pdf}{solutions}
\item
  \href{exam_files/midterm2022.pdf}{2022},
  \href{exam_files/midterm2022_solns.pdf}{solutions}
\item
  \href{exam_files/midterm2021.pdf}{2021},
  \href{exam_files/midterm2021_solns.pdf}{solutions}
\end{itemize}

\subsection{Previous finals}\label{previous-finals}

Covers all material, with a focus on the remaining lectures and labs.

\begin{itemize}
\tightlist
\item
  \href{exam_files/final2024.pdf}{2024},
  \href{exam_files/final2024_solns.pdf}{solutions}
\item
  \href{exam_files/final2022.pdf}{2022},
  \href{exam_files/final2022_solns.pdf}{solutions}
\item
  \href{exam_files/final2021.pdf}{2021},
  \href{exam_files/final2021_solns.pdf}{solutions}
\end{itemize}

\section{General info}\label{general-info}

\subsection{Land acknowledgement}\label{land-acknowledgement}

I wish to acknowledge this land on which the University of Toronto
operates. For thousands of years it has been the traditional land of the
Huron-Wendat, the Seneca, and the Mississaugas of the Credit. Today,
this meeting place is still the home to many Indigenous people from
across Turtle Island and I am grateful to have the opportunity to work
on this land. For more information see
\href{https://indigenous.utoronto.ca/about/land-acknowledgement/}{University
of Toronto's land acknowledgement}.

\subsection{Group norms}\label{group-norms}

The University of Toronto is committed to equity, human rights, and
respect for diversity. All members of the learning environment in this
course should strive to create an atmosphere of mutual respect where all
members of our community can express themselves, engage with each other,
and respect one another's differences. U of T does not condone
discrimination or harassment against any persons or communities. Please
contact me if you have any concerns. For more information see the
\href{https://governingcouncil.utoronto.ca/secretariat/policies/code-student-conduct-december-13-2019}{Code
of Student Conduct}.

\subsection{Accessibility}\label{accessibility}

The University provides academic accommodations for students with
disabilities in accordance with the terms of the
\href{http://www.ohrc.on.ca/en/ontario-human-rights-code}{Ontario Human
Rights Code}. This occurs through a collaborative process that
acknowledges a collective obligation to develop an accessible learning
environment that both meets the needs of students and preserves the
essential academic requirements of the University's courses and
programs. Students with diverse learning styles and needs are welcome in
this course. If you have a disability that may require accommodations,
please feel free to get in touch with me and/or the
\href{https://studentlife.utoronto.ca/department/accessibility-services/}{Accessibility
Services office}.

\subsection{Religious observances}\label{religious-observances}

The University provides reasonable accommodation of the needs of
students who observe religious holy days other than those already
accommodated by ordinary scheduling and statutory holidays. Students
have a responsibility to alert members of the teaching staff in a timely
fashion to upcoming religious observances and anticipated absences and I
will make every reasonable effort to avoid scheduling tests,
examinations or other compulsory activities at these times. Please reach
out to me as early as possible to communicate any anticipated absences
related to religious observances, and to discuss any possible related
implications for course work.

\subsection{Family care
responsibilities}\label{family-care-responsibilities}

The University of Toronto strives to provide a family-friendly
environment. You may wish to inform me if you are a student with family
responsibilities. If you are a student parent or have family
responsibilities, you also may wish to visit the
\href{https://familycare.utoronto.ca}{Family Care Office website}.

\section{Resources}\label{resources}

There are many resources available at the University of Toronto to help
you succeed in this course. Below are a few:

\begin{itemize}
\tightlist
\item
  \href{https://writing.utoronto.ca/}{Writing Center}
\item
  \href{https://www.academicintegrity.utoronto.ca/}{Academic integrity}
\item
  \href{https://studentlife.utoronto.ca/wp-content/uploads/SLC8581_7-Grandfathers-in-Academic-Integrity-AODA.pdf}{More
  on academic integrity}
\item
  \href{https://teaching.utoronto.ca/teaching-support/u-of-t-resources/teaching/students/}{CTSI
  list of supports}
\item
  \href{https://q.utoronto.ca/enroll/ALEYMP}{Academic success module}
\item
  \href{https://q.utoronto.ca/courses/46670/pages/student-guide}{Get
  help with Quercus}
\end{itemize}

\end{document}
